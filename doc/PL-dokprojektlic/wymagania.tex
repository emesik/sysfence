\section{Wymagania}

Do�o�yli�my stara�, by mechanizmy obs�ugi proces�w, sygna��w i
metod IPC (semafory i pami�� dzielona) korzysta�y ze standardowych funkcji
dost�pnych w ka�dym wsp�czesnym systemie uniksowym.

Ponadto, dla prawid�owego dzia�ania, sysfence wymaga od systemu:
\begin{itemize}
    \item systemu plik�w \emph{proc} zamontowanego w katalogu \texttt{/proc}
    (w przypadku u�ycia innej �cie�ki konieczna jest modyfikacja odpowiedniej
    sta�ej przed kompilacj�),
    \item funkcji systemowej \texttt{statfs()}
    %
    %
    %   LSB opisuje statfs() jako 'deprecated'
    %   jesli zdazymy, zamienmy je na statvfs(), zgodne z tym standardem
    %
    %
\end{itemize}

W chwili obecnej program jest przeznaczony dla systemu operacyjnego Linux.
Poniewa� w trakcie tworzenia aplikacji nie dysponowali�my maszynami
dzia�aj�cymi pod innymi systemami, nie mogli�my przetestowa� kompilacji i
zachowania aplikacji w innym �rodowisku. Powy�sze wymagania nie s� jednak
wyg�rowane i przewidujemy, �e sysfence bez problem�w mo�e zosta� uruchomiony
na innych darmowych platformach uniksowych, jak np. FreeBSD, OpenBSD, NetBSD
czy Hurd.
